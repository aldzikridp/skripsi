\documentclass[./bab_1.tex]{subfiles}
\begin{document}
\section{Latar Belakang}
\paragraph*{}Perkembangan koneksi internet yang semakin
cepat dan terjangkau mengubah cara bagaimana dunia bekerja.
Dari bidang komunikasi dan hiburan, hingga bidang komersial
dan pendidikan, internet telah menjadi hal yang umum di
masyarakat modern. Di ranah bisnis, internet menciptakan
peluang baru bagi perusahaan untuk menjangkau pasar yang
lebih luas dan beroperasi dengan lebih
efisien dengan cara yang inovatif(\cite{parinda23}). Oleh
karena itu perusahaan perlu merangkul dan mengintegrasikan
internet ke dalam infrastruktur yang dimiliki sebagai sarana agar
tetap unggul dalam ranah bisnis yang terus berkembang.

\paragraph*{}PT.Pundi Mas Berjaya adalah salah satu penyedia
solusi perangkat lunak di pasar global yang memberikan
solusi bisnis mengadopsi teknologi informasi terkini. Solusi
berbasis layanan untuk para pelanggan yang tersebar di
beberapa negara dan memiliki model pengembangan on-site dan
off-site. Perusahaan ini merancang dan mengembangkan banyak
solusi di bidang properti, otomotif, transportasi,
pengirimanan makanan, pengiriman barang dan ecommerce.

\paragraph*{}Tentunya dengan memiliki klien yang banyak dan
berasal dari perusahaan yang beragam PT.Pundi Mas Berjaya
memiliki masalah tersendiri. Setiap perusahaan memiliki
\textit{requirements} dan \textit{tools} masing-masing yang
unik untuk memenuhi kebutuhan infrastruktur IT perusahaan.
Dengan begitu tiap-tiap perangkat lunak yang dibuat dan
di-\textit{host} oleh PT.Pundi Mas Berjaya membutuhkan
\textit{requirements} yang berbeda pula. Misalnya, perangkat
lunak untuk perusahaan A menggunakan LAMP \textit{stack} dengan PHP versi 7, sedangkan
perangkat lunak untuk perusahaan B menggunakan LAMP
\textit{stack} dengan PHP versi 8.
Selain itu tiap layanan yang di-\textit{host} harus dengan
mudah direplikasi untuk membuat \textit{instance} baru,
contohnya membuat \textit{instance} email server baru untuk
pelanggan baru.
Sehingga PT.Pundi Mas Berjaya perlu mengisolasi tiap-tiap
perangkat lunaknya sehingga dapat bekerja dengan baik dan
menghindari \textit{conflict} antar versi serta dapat
disiapkan dan dijalankan dengan cepat.

\paragraph*{}Maka dari itu, PT.Pundi Mas Berjaya
mengimplementasikan \textit{containerization} menggunakan
docker dengan tujuan untuk memaketkan dan mengisolasi
perangkat lunaknya, mengotomatisasikan \textit{deployment}
dan manajemen aplikasi, serta memastikan perangkat lunak
dapat dijalankan di komputer yang berbeda.

\end{document}


