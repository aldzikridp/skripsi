\documentclass[./bab_1.tex]{subfiles}
\begin{document}
\section{Latar Belakang}
  \paragraph*{} Hampir seluruh perangkat saat ini terhubung
  dengan internet, seperti \textit{smartphone}, komputer,
  dan \textit{smart TV}. Meskipun tekonologi internet
  sangatlah bermanfaat, tetapi teknologi ini juga
  menimbulkan ancaman keamanan.

  \paragraph*{} Dari survey yang dilakukan, mayoritas
  developer tidak ragu melakukan \textit{personalized-ads}.
  Jenis iklan ini menggunakan data pengguna untuk memilihkan
  iklan yang dianggap relevan berdasarkan data pengguna. Hal
  ini tidak mengherankan karena jenis iklan ini menawarkan
  pendapatan yang lebih tinggi dibandingkan iklan biasa,
  karena peluang pengguna membuka iklan yang ditawarkan
  lebih besar(\cite{tahaei2021}). Penelitian lain juga
  menunujukkan bahwa 77\% aplikasi Android gratis
  menggunakan \textit{ad library}(\cite{he2018, jin2021}).

  \paragraph*{} Iklan-iklan yang ditampilkan sering kali
  sangat intrusif dan menggangu. Contohnya seperti iklan
  yang menutupi konten, iklan melalui \textit{pop-up}, atau
  iklan yang otomatis mengarahkan pengunjung ke \textit{tab}
  baru yang memuat iklan. Atau bahkan menampilkan iklan
  yang menjurus ke arah konten dewasa.

  \paragraph*{} Masalah lain selain iklan yang mengganggu,
  adalah masalah keamanan. Banyak layanan yang memasang
  iklan juga menggunakan layanan \textit{web analytics}
  untuk melakukan \textit{tracking} terhadap pengunjung atau
  penggunanya. Layanan web analytics inilah yang dapat
  menjadi celah keamanan. Contohnya pada tahun 2019, layanan
  Google Analytics digunakan untuk mengiklankan website
  phising(\cite{charlie2019}). Sedangkan di tahun 2020,
  peretas memasukkan kode berbahaya ke dalam website yang
  diretas, yang mana kode tersebut mengumpulkan informasi
  kredit pengguna dan mengirimkannya menggunakan Google
  Analytics, kemudian peretas akan mengakses data kredit
  yang dikumpulkan di akun Google Analytics
  miliknya(\cite{ravie2020}).

  \paragraph*{}Hal-hal tersebut bisa dicegah dengan
  menggunakan metode DNS \textit{ad-blocking}, dengan metode
  ini, kita bisa melakukan pemblokiran domain yang melakukan
  \textit{tracking}, domain yang digunakan sebagai
  tempat menyimpan kode JavaScript yang berbahaya, atau
  domain milik penyedia iklan. Tetapi metode
  ini hanya bisa melakukan pemblokiran berdasarkan nama
  domain, sehingga jika iklan atau kode JavaScript berada
  pada domain yang sama dengan website utama, maka kita
  harus mengizinkan iklan, dan kode JavaScript berjalan,
  atau memblokir domain website tersebut sehingga tidak bisa
  diakses.

  \paragraph*{}Metode lain adalah menggunakan
  \textit{add-on} peramban, seperti \textit{UBlock Origin}.
  Metode ini mampu melakukan \textit{blocking} terhadap
  iklan atau kode JavaScript yang berbahaya, walaupun
  keduanya berada pada domain yang sama dengan website
  utama. Sayangnya \textit{add-on} ini hanya bisa
  di-\textit{install} pada peramban tertentu, seperti
  Mozilla Firefox, dan Chrome Desktop.

  \paragraph*{}Berdasarkan permasalahan yang disebutkan
  di atas, tujuan saya adalah membangun sistem HTTPS
  filtering dengan menggunakan Mitmproxy. Sistem ini
  bertujuan untuk menyaring konten yang tidak diinginkan
  agar tidak diakses.

  \paragraph*{}Filtering dengan menggunakan Mitmproxy saya
  pilih karena tools ini merupakan proxy, sehingga bisa
  digunakan di banyak perangkat seperi metode DNS
  \textit{ad-blocking}, tetapi bisa di buat sedemikian rupa
  dengan menuliskan \textit{script} Python sehingga mampu
  melakukan filtering seperti \textit{add-on} pada browser,
  atau bahkan melebihinya.

  \paragraph*{}Selain itu kelebihan Mitmproxy yang lain
  adalah software ini bersifat \textit{open source} dan
  gratis. Software ini bisa digunakan sebagai alat untuk
  mencegat, inspeksi, dan memodifikasi lalu lintas web,
  seperti HTTP/1, HTTP/2, WebSockets, atau protokol lainnya
  yang dilindungi oleh SSL/TLS.
\end{document}


