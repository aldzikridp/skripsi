\documentclass[./bab_1.tex]{subfiles}
\begin{document}
\section{Sistematika Penulisan}
\paragraph*{}Berikut merupakan sistematika penilisan skripsi
yang akan dibuat:

\begin{itemize}
  \item\textbf{Bab I Pendahuluan}\\
  Pada bab ini berisi tentang penjelasan
  mengenai latar belakang masalah, rumusan masalah, ruang
  lingkup, tujuan penelitian, manfaat penelitian, dan
  sistematika penulisan

  \item\textbf{Bab II Tinjauan Pustaka dan Dasar Teori}\\
  Pada bab ini berisi tentang pembahasan sumber pustaka yang
  digunakan sebagai pedoman perancangan penelitian yang
  berhubungan dengan penelitian yang sedang diteliti.

  \item\textbf{Bab III Metode Penelitian}\\
  Pada bab ini berisi tentang pembahasan analisis kebutuhan,
  bahan/data, dan juga perancangan sistem yang akan
  digunakan.

  \item\textbf{Bab IV Implementasi dan Pembahasan}\\
  Pada bab ini menguraikan tentang pembuatan aplikasi yang
  merupakan implementasi dari hasil analisa dan perancangan,
  pengujian sistem, dan juga kesimpulan.

  \item\textbf{Bab V Penutup}\\
  Pada bab ini berisi kesimpulan dari hail pembahasan
  penerapan sistem dan saran-saran guna pengembang sistem
  yang telah dibuat.
\end{itemize}
\end{document}
