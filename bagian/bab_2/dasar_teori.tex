\documentclass[./bab_2.tex]{subfiles}
\begin{document}
\bagian{Dasar Teori}
\begin{enumerate}[label=\textbf{\arabic*.}]
  \item \textbf{Mitmproxy}
  \paragraph*{}Mitmproxy adalah seperangkat alat yang
  menyediakan proxy yang mampu mencegat SSL/TLS untuk
  protokol HTTP/1, HTTP/2, dan WebSockets(\cite{mitm}).

  \paragraph*{}Fitur dari Mitmproxy antara lain mencegat
  \textit{request} dan \textit{response} dari  HTTP dan
  HTTPS dan memodifikasinya, Kemampuan untuk mengubah lalu
  lintas HTTP dengan menggunakan \textit{script} Python,
  dll(\cite{mitm}).

  \item \textbf{HTTPS}
  \paragraph*{}HTTPS (Hypertext Transfer Protocol Secure)
  adalah protokol komunikasi internet yang melindungi
  integritas dan kerahasiaan data antara komputer pengguna
  dan situs. Pengguna menginginkan pengalaman online yang
  aman dan bersifat pribadi saat menggunakan situs.

  \paragraph*{}Data yang dikirim menggunakan HTTPS diamankan
  melalui protokol Transport Layer Security (TLS), yang
  memberikan tiga lapis perlindungan utama:
  \begin{enumerate}
    \item Enkripsi\\
      Mengenkripsi data pertukaran untuk menjaga keamanannya
      dari penyadap. Artinya, saat pengguna menjelajahi
      situs, tidak ada yang dapat "menguping" percakapan,
      melacak aktivitas di berbagai halaman, atau mencuri
      informasi mereka.
    \item Integritas data\\
      Data tidak dapat diubah atau dirusak selama transfer,
      dengan sengaja atau tidak, tanpa terdeteksi.
    \item Autentikasi\\
      Membuktikan bahwa pengguna Anda berkomunikasi dengan
      situs yang diinginkan. Protokol tersebut melindungi
      dari serangan \textit{man in the middle} dan membangun
      kepercayaan pengguna, yang dapat memberikan keuntungan
      lain untuk bisnis Anda.
  \end{enumerate}
  (\cite{googledev-https})

  \item \textbf{Web Content Filtering}
    \paragraph*{}Web content filtering merupakan saringan
    konten website yang digunakan oleh perorangan, kelompok,
    maupun organisasi untuk melakukan penyaringan terhadap
    situs-situs yang tidak diperbolehkan oleh pihak berwenang
    maupun yang tidak berhubungan dengan tujuan bisnis atau
    organisasi agar tidak dapat diakses(\cite{dewi2021}).

  \item \textbf{Proxy Server}
     \paragraph*{}Proxy server (peladen proxy) adalah sebuah
     komputer server atau program komputer yang dapat bertindak
     sebagai komputer lainnya untuk melakukan request terhadap
     content dari Internet atau Intranet(\cite{rendra2013}).
    
     \paragraph*{}Proxy server bertindak sebagai gateway
     terhadap dunia ini Internet untuk setiap komputer clien.
     Proxy server tidak terlihat oleh komputer client: seorang
     pengguna yang berinteraksi dengan Internet melalui sebuah
     proxy server tidak akan mengetahui bahwa sebuah proxy
     server sedang menangani request yang dilakukannya. Web
     server yang menerima request dari proxy server akan
     menginterpretasikan request-request tersebut seolah-olah
     request itu datang secara langsung dari komputer client,
     bukan dari proxy server(\cite{rendra2013}).

\end{enumerate}
\end{document}
