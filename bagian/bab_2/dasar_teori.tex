\documentclass[./bab_2.tex]{subfiles}
\begin{document}
\section{Dasar Teori}
%\begin{enumerate}[label=\textbf{\arabic*.}]
  \subsection{PT.Pundi Mas Berjaya}
    \paragraph*{} PT.Pundi Mas Berjaya adalah salah satu
    penyedia solusi perangkat lunak di pasar global yang
    memberikan solusi bisnis mengadopsi teknologi informasi
    terkini. Solusi berbasis layanan untuk para pelanggan
    yang tersebar di beberapa negara dan memiliki model
    pengembangan on-site dan off-site. Sejak berdiri pada
    tahun 2014, Perusahaan telah merancang, mengembangkan
    dan digunakan banyak solusi di bidang properti,
    otomotif, transportasi, pengirimanan makanan, pengiriman
    barang dan ecommerce.

    \paragraph*{} PT.Pundi Mas Berjaya memiliki tim yang
    kuat dengan ukuran pelaksanaan proyek besar dan
    mengerahkan seluruh kemampuannya untuk memberikan
    pelayanan yang berkualitas tinggi kepada pelanggan.
    Salah satu kunci keberhasilan PT.Pundi Mas Berjaya
    adalah adaptasi perusahaan tersebut
    terhadap berbagai macam dan aneka kebutuhan akan sistem
    informasi komputer sehingga menghasilkan Manajemen
    Kualitas, Manajemen Proyek, kebutuhan Infrastruktur, dan
    lainnya dengan sempurna untuk memberikan kepuasan kepada
    para pelanggan PT.Pundi Mas Berjaya. Pelayanan PT.Pundi Mas Berjaya sifatnya terpadu
    menggunakan media responsif website, mobile website,
    mobile apps untuk mendukung aktifitas yang lebih
    optimal.

    PT.Pundi Mas Berjaya memiliki tim berpengalaman di
    bidang:
    \begin{enumerate}
      \item Manajemen teknis website
      \item Manajemen teknis mobile application
      \item Manajemen bisnis untuk aplikasi yang sedang dibangun
      \item Manajemen kendali mutu untuk aplikasi yang dijalankan
    \end{enumerate}
    
    Sedangkan fokus bisnis dari perusahaan ini antara lain:
    \begin{enumerate}
      \item Aplikasi properti online kolaborasi dalam dan luar negeri
      \item Aplikasi transportasi berbasis aplikasi
      \item Aplikasi bank data untuk pebisnis
      \item Aplikasi pelayanan pengantaran makanan
      \item Aplikasi pelayanan pengiriman bunga/parcel
      \item Aplikasi Deal untuk aneka voucher
      \item Aplikasi e-commerce
    \end{enumerate}

  \subsection{Container}
    \paragraph*{}Menurut \textcite{shah19}
    \textit{Containerization} adalah cara untuk menjalankan
    beberapa perangkat lunak yang berbeda di satu mesin.
    Yang masing-masingnya berjalan di dalam lingkungan yang
    terisolasi yang disebut sebagai \textit{container}.
    \textit{Container} adalah lingkungan yang terisosali
    untuk perangkat lunak. Container mengikat semua berkas
    dan \textit{library} yang dibutuhkan oleh aplikasi untuk
    berfungsi dengan baik.

  \subsection{Docker}
    \paragraph*{}Docker adalah sebuah project
    \textit{open-source} yang menyediakan platform terbuka
    untuk developer maupun sysadmin untuk dapat membangun,
    mengemas, dan menjalankan aplikasi dimanapun di dalam
    sebuah container(\cite{docker_overview}).

    \paragraph*{}Docker menggunakan arsitektur client-server. Dimana client
    dan docker berkomunikasi dengan daemon docker, yang
    melakukan suatu tindakan untuk membangun,
    menjalankan, dan mendistribusikan container
    docker. Client docker dan daemon dapat berjalan pada
    sistem yang sama. Client docker dan daemon
    berkomunikasi menggunakan REST API, melalui soket
    UNIX atau antarmuka jaringan(\cite{docker_overview}).

  \subsection{Docker Compose}
    \paragraph*{}Docker Compose adalah alat untuk mendefinisikan
    dan menjalankan aplikasi Docker multi-kontainer. Dengan
    Docker Compose, file YAML bisa digunakan untuk mengonfigurasi
    layanan aplikasi. Kemudian, hanya dengan satu perintah, Docker
    akan membuat dan memulai semua layanan dari konfigurasi
    YAML tersebut.

  \subsection{LAMP \textit{stack}}
    \paragraph*{}LAMP \textit{stack} adalah gabungan dari empat
    teknologi perangkat lunak berbeda yang digunakan oleh
    developer untuk membangun situs web dan aplikasi web.
    LAMP adalah singkatan dari sistem operasi Linux; server
    web Apache; server basis data MySQL; dan bahasa
    pemrograman PHP. Keempat teknologi ini bersifat
    \textit{open-source}, yang berarti LAMP dikelola komunitas dan
    tersedia secara bebas untuk digunakan oleh siapa saja.
    Developer menggunakan LAMP \textit{stack} untuk membuat,
    meng-host, dan memelihara konten web. LAMP
    \textit{stack}
    adalah solusi populer yang mendukung banyak situs web
    yang biasa gunakan saat ini(\cite{amazon_lamp}).

  \subsection{Email Server}
    \paragraph*{} Email server adalah perangkat lunak yang
    berfungsi untuk mengirim dan menerima email. Email
    server biasanya menjadi seebutan untuk \textit{Mail
    Transfer Agent} (MTA) dan \textit{Mail Delivery Agent}
    (MDA), yang keduanya memiliki fungsi yang
    berbeda(\cite{cf_email}).

    \paragraph*{} Pesan email dikirim dan diterima dengan
    menggunakan dua jenis email server: \textit{outgoing
    mail server} atau \textit{Mail Transfer Agent}, dan
    \textit{incoming mail server} atau \textit{Mail Delivery
    Agent}. MTA akan menerima pesan yang keluar dari
    \textit{email client} pengirim, kemudian mengirimkannya
    ke MDA, yang mana bertanggung jawab untuk menyimpan
    sementara dan menyampaikan pesan ke \textit{email
    client} penerima(\cite{cf_email}).

    \subsection{DNS TXT Record}
    TXT Record, Merupakan sebuah record DNS yang digunakan
    untuk menyimpan informasi berupa text pada domain. Teks
    pada TXT Records harus diformat sesuai dengan teknologi
    yang akan gunakan(\cite{rfc1464}).

    \subsection{DKIM}
    \paragraph*{} Merujuk pada dokumen RFC 6376, DKIM
    \textit{(DomainKeys Identified Mail)} adalah teknik untuk
    memverifikasi keaslian pesan email dengan memeriksa
    tanda tangan digital yang disematkan di header pesan.
    Tanda tangan ini dihasilkan dengan mengenkripsi hash
    konten pesan dengan \textit{private key}, yang dapat
    diverifikasi oleh penerima menggunakan \textit{public
    key} yang diterbitkan dalam data DNS TXT domain
    pengirim(\cite{rfc6376}).

    \paragraph*{} Dengan DKIM email server penerima dapat
    memverifikasi bahwa pesan yang masuk tidak diubah
    saat transit dan berasal dari domain yang digunakan.
    DKIM \textit{signature} dapat dibuat dengan menambahkan
    header ke pesan yang berisi informasi tentang pesan dan
    tanda tangan kriptografi dari isi pesan dan beberapa
    header. Server email penerima kemudian dapat menggunakan
    informasi di header DKIM dan kunci publik yang
    dipublikasikan di DNS domain pengirim untuk
    memverifikasi keaslian dan integritas pesan(\cite{rfc6376}).

    \subsection{SPF}
    \paragraph*{} Menurut dokumen RFC 7208, SPF \textit{(Sender
    Policy Framework)} adalah protokol autentikasi email yang
    dirancang untuk mendeteksi dan mencegah spoofing email
    dengan cara memverifikasi identitas domain yang diklaim
    berasal dari email. SPF bekerja dengan memberikan daftar
    alamat IP atau nama host pengiriman resmi untuk domain
    di DNS, dan server email penerima dapat menggunakan
    informasi tersebut untuk memvalidasi keaslian pesan
    masuk(\cite{rfc7208}).

    \paragraph*{} Saat email diterima, server email penerima
    memeriksa data SPF untuk domain pada header "From" pesan
    untuk menentukan apakah pengirim diizinkan untuk
    mengirim email atas nama domain yang digunakan. Jika
    server pengirim tidak tercantum dalam catatan SPF, pesan
    tersebut mungkin akan ditandai sebagai spam atau bahkan
    langsung ditolak(\cite{rfc7208}).

    \subsection{DMARC}
    \paragraph*{} Menurut dokumen RFC 7489, DMARC
    \textit{(Domain-based Message Authentication, Reporting \&
    Conformance)} merupakan metode autentikasi, pengaturan
    dan pelaporan yang menggunakan
    kombinasi metode autentikasi SPF (Sender Policy
    Framework) dan DKIM (DomainKeys Identified Mail) untuk
    menentukan apakah pesan email itu sah. Dengan
    menggunakan DMARC, email server penerima dapat
    mengambil tindakan yang sesuai, seperti menolak atau
    mengkarantina pesan, berdasarkan peraturan yang
    dipublikasikan pemilik domain di
    DMARC-nya(\cite{rfc7489}).

    DMARC juga memiliki mekanisme pelaporan bagi pemilik
    domain untuk menerima umpan balik tentang pesan yang lulus
    atau gagal dalam evaluasi DMARC, dan membantu pemilik
    domain memantau pengiriman email dan mendeteksi potensi terjadinya
    spoofing email(\cite{rfc7489}).
%\end{enumerate}
\end{document}
