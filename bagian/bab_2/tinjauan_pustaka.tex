\documentclass[./bab_2.tex]{subfiles}
\begin{document}
\section{Tinjauan Pustaka}
  \paragraph*{}Dalam mengimplementasikan LAMP \textit{stack}
  dan email server dengan menggunakan Docker, sebagai
  pedoman dan pembanding maka digunakan beberapa pustaka
  yang berkaitan dengan \textit{Docker}. Pustaka yang
  digunakan sebagai rujukan antara lain:

  \DefTblrTemplate{caption-tag}{default}{\textbf{Tabel\hspace{0.25em}\thetable}}
  \DefTblrTemplate{contfoot-text}{default}{}
  \DefTblrTemplate{conthead-text}{default}{}
  \DefTblrTemplate{middlehead,lasthead}{default}{}
  \begin{footnotesize}
    \begin{singlespace}
  \begin{longtblr}[caption= {Tabel Pustaka}]{hlines, vlines,
    column{1}={0.20\linewidth}, column{2}={0.20\linewidth},
    column{3}={0.35\linewidth}, column{4}={0.10\linewidth},
    rowhead=1} 
  \textbf{Penulis}   & \textbf{Topik} &
    \textbf{Hasil Penelitian} & \textbf{Teknologi yang digunakan}\\
    \textcite{wij18} & \citetitle{wij18} & {Menghasilkan aplikasi lowongan kerja yang
    dibuat menjadi docker images yang dapat berjalan di
    container docker sehingga dapat dimanfaatkan di penyedia
    \textit{platform as a service} (PAAS).} & {Docker}\\

    \textcite{yatno20} & \citetitle{yatno20} &
    {Mengimplementasikan  sistem  virtualisasi server
    berbasis docker container di SMK Negeri 1 Rangkasbitung
    dan pengujian penggunaan sumber daya dari penggunaan
    Docker sebagai sistemt virtualisasi} & {Docker, Nginx,
    Apache Web Server, Apache Jmeter}\\

    \textcite{mega2021} & \citetitle{mega2021} & {Menghasilkan
    sistem untuk membantu mahasiswa melakukan praktek
    pemrograman pada saat pembelajaran jarak jauh. Dengan sistem ini
    mahasiswa dapat mengerjakan tugas di beragam perangkat
    dan memudahkan dosen memeriksa hasil praktek.} &
    {Docker, Jupyter Notebook}\\

    \textcite{setyo21} & \citetitle{setyo21} & {Analisa performa
    microservice yang diimplementasikan di docker container.
    Dalam penelitian ini disimpulkan docker memiliki
    efisiensi yang lebih baik dibanding virtualisasi
    hypervisor.} & {Docker, Apache Jmeter}\\

    \textcite{furnama22} & \citetitle{furnama22} & {Menghasilkan
    sistem \textit{backend} pembayaran terintegrasi Midtrans dengan
    arsitektur microservice yang berjalan di atas docker.} &
    {Docker, Midtrans}\\

    \textcite{putra21} & \citetitle{putra21} & {Penelitian ini
    menghasilkan sistem cloud storage menggunakan OwnCloud
    dan Docker. Dengan menggunakan Docker, pemasangan
    OwnCloud tidak mempengaruhi sistem \textit{host}, selain
    itu Docker juga menghemat penggunaan resource.} &
    {Docker, OwnCloud}\\

    \textcite{kris22} & \citetitle{kris22} & {Pada penelitian ini
    menghasilkan sistem web hosting dengan menggunakan
    Wordpress sebagai CMS yang memanfaatkan nginx sebagai
    web servernya, sedangkan untuk reverse proxy digunakan
    Apache Web Server.} & {Docker, Wordpress, Nginx} \\

    {Aldzikri Dwijayanto Prathama (Diusulkan)} & {Implementasi Lamp Stack Dan Email Server
    Menggunakan Docker di PT.Pundi Mas Berjaya (Magang Bersertifikat Kampus
    Merdeka)} &{Pada
    penelitian ini adalah
    membahas cara membangun sebuah server LAMP dan email,
    dengan metode \textit{containerization} menggunakan
    Docker. Sehingga memudahkan manajemen, dan
    \textit{deployment} aplikasi.} & {Docker, LAMP-stack,
    Postfix, Dovecot}\\

    \end{longtblr}
    \end{singlespace}
  \end{footnotesize}
  
\end{document}
