\documentclass[./bab_2.tex]{subfiles}
\begin{document}
\section{Tinjauan Pustaka}
  \paragraph*{}Dalam mengimplementasikan MITM-Proxy untuk
  HTTPS \textit{filtering}, sebagai pedoman dan pembanding
  maka digunakan beberapa pustaka yang berkaitan dengan
  \textit{ad-blocking}. Pustaka yang digunakan sebagai
  rujukan antara lain:

  \paragraph*{} Dari penelitian yang dilakukan oleh
  \cite{uni2021}, yang berjudul \citetitle{uni2021}.
  Tujuan dari dilakukannya penelitian ini adalah untuk
  mengetahui rancangan sistem \textit{blocking} situs
  berbahaya dengan menggunakan \textit{pi-hole} berbasis
  docker dan \textit{openvpn}, dan menguji kemampuan
  \textit{blocking} situs dan performa dari \textit{pihole}
  dalam \textit{docker container}.

  \paragraph*{} \cite{yusoff2020}, Melakukan penelitian yang
  berjudul \textit{\citetitle{yusoff2020}}. Penelitian
  tersebut bertujuan untuk membangun server openvpn
  menggunakan Raspberry-pi dan menggunakan pi-hole sebagai
  sistem \textit{ad-block}. Selain itu penelitian ini juga
  memanfaatkan kemampuan pi-hole memahami \textit{regular
  expression} sebagai \textit{parental control} untuk
  mencegah diaksesnya situs-situs dewasa.

  \paragraph*{} Prosiding dengan judul
  \textit{\citetitle{sarath2020}}yang ditulis oleh
  \cite{sarath2020}. Tujuan dari penelitian tersebut adalah
  membuat sistem keamanan jaringan yang murah untuk bisnis
  kelas menengah ke bawah, dengan menggunakan Raspberry-pi
  sebagai perangkat kerasnya. Sedangkan di bagian perangkat
  lunak, digunakan Dnsmasq. Dnsmasq digunakan sebagai DNS
  \textit{filtering} dari domain yang dianggap berbahaya.

  \paragraph*{} Prosiding oleh \cite{wahyudi2020}, dengan
  judul \citetitle{wahyudi2020}. Penelitian ini membahas
  bagaimana cara mengimplementasikan pi-hole di Ubuntu
  server sebagai DNS \textit{blocking} pada jaringan SMK TIK
  Darussalam Medan.

  \paragraph*{} Dari penelitian yang dilakukan oleh
  \cite{habibi2022} dengan judul \citetitle{habibi2022}.
  Penelitian ini membahas tentang pemasangan infrastruktur
  jaringan internet dengan menggunakan \textit{Point to
  Multipoint}, meminimalkan hilangnya paket data pada saat
  transmisi menggunakan RED dan PCQ dan menggunakan Sistem
  pi-hole di server untuk meminimalkan iklan yang muncul di
  protokol UDP.

  %\newpage

  \begin{longtable}{|>{\hspace{0pt}}m{0.15\linewidth}|>{\hspace{0pt}}m{0.25\linewidth}|>{\hspace{0pt}}m{0.10\linewidth}|>{\hspace{0pt}}m{0.35\linewidth}|} 
  \hline
  \textbf{Penulis}   & \textbf{Judul Penelitian} & \textbf{Tools}               & \textbf{Hasil}                                                                                                                                                                                                                                                \endfirsthead 
  \hline
  \cite{uni2021}     & \citetitle{uni2021}       & pi-hole, OpenVPN, dan docker & Server vpn dengan menggunakan protokol OpenVPN, yang menggunakan pi-hole sebagai sistem blocking domain berbahaya. Server tersebut berjalan di dalam docker container.                                                                                        \\ 
  \hline
  \cite{yusoff2020}  & \citetitle{yusoff2020}    & pi-hole dan raspberry pi     & DNS Filtering dengan menggunakan pi-hole yang berjalan di atas raspberry pi. Selain itu dengan menggunakan regular expression, pi-hole dapat mengenali situs-situs dewasa sehingga dapat menjadi solusi parental control.                                     \\ 
  \hline
  \cite{sarath2020}  & \citetitle{sarath2020}    & Dnsmasq                      & Server DNS dengan menggunakan Dnsmasq yang berjalan di atas raspberry pi. Sistem ini dapat menjadi pilihan untuk mencegah client mengakses domain yang berbahaya. Sistem ini dapat menjadi pilihan untuk diimplementasikan di usaha kelas menengah ke bawah.  \\ 
  \hline
  \cite{wahyudi2020} & \citetitle{wahyudi2020}   & pi-hole, ubuntu server       & Server DNS pi-hole yang berjalan di atas sistem operasi ubuntu server. Sistem ini diimplementasikan oleh penulis di jaringan SMK TIK Darussalam Medan.                                                                                                        \\ 
  \hline
  \cite{habibi2022}  & \citetitle{habibi2022}    & pi-hole                      & DNS filtering diimplementasikan di jaringan sehingga mengurangi penggunaan data pada jaringan internet warga, sehingga jaringan tersebut lebih optimal.                                                                                                       \\
  \hline
  \end{longtable}
  
  %\newpage

  \paragraph*{} Penelitian yang sudah dilakukan di atas,
  memiliki tujuan yang serupa dengan penelitian ini yaitu
  \textit{ad-blocking}. Namun di penelitian ini, yang
  membahas implementasi MITM-Proxy untuk https filtering,
  menggunakan MITM-Proxy sebagai perangkat lunak utamanya.
  
  \paragraph*{} Perbedaan lainnya dari penelitian yang sudah
  dilakukan adalah tujuan dari penelitian ini adalah
  membangun sistem yang tidak hanya menyaring berdasarkan
  domain, tetapi juga menyaring berdasarkan konten yang
  diakses. Konten yang dimaksud adalah seperti file HTML,
  CSS, dan JavaScript yang utamanya digunakan untuk
  menampilkan web, juga respon dari server seperti JSON yang
  utamanya digunakan di \textit{API}.
\end{document}
