\documentclass[./bab_3.tex]{subfiles}
\begin{document}
  \section{Analisis Kebutuhan Perangkat}
  \paragraph*{}Dalam proses pembuatan sistem ini, membutuhkan
  perangkat lunak dan perangkat keras yang digunakan
  sebagai \textit{server} dan \textit{client} untuk
  me-\textit{remote} \textit{server}, analisis kebutuhan dalam
  pembuatan sistem ini antara lain:
  
  \subsection{Kebutuhan Perangkat Lunak}
  \begin{enumerate}
    \item \textbf{Perangkat Lunak Server}
      \begin{enumerate}
        \item Sistem Operasi Ubuntu Server
        \item OpenSSH
        \item Docker
        \item Rsync
        \item Vim
      \end{enumerate}
    \item \textbf{Perangkat Lunak Client}
      \begin{enumerate}
        \item Sistem Operasi NixOS
        \item OpenSSH
        \item Rsync
        \item Neovim
      \end{enumerate}
  \end{enumerate}

  \subsection{Kebutuhan Perangkat Keras}
  \begin{enumerate}
    \item \textbf{Perangkat Keras Server}
      \begin{enumerate}
        \item Dell PowerEdge Rack Server
        \item Switch
        \item Router
      \end{enumerate}
    \item \textbf{Perangkat Keras Client}
      \begin{enumerate}
        \item Laptop Lenovo ThinkPad T440p
        \item CPU i7 4702MQ
        \item RAM 8GB
        \item SSD 250GB
      \end{enumerate}
  \end{enumerate}

  \subsection{Kebutuhan Lainnya}
  \begin{enumerate}
    \item Domain Name
  \end{enumerate}

\end{document}
