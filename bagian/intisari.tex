\documentclass[../../SKRIPSI_ALDZIKRI_DWIJAYANTO_PRATHAMA.tex]{subfiles}
\begin{document}
%\chapter*{\filcenter{\Large{Intisari}}}
\chapter*{INTISARI}\vspace*{2ex}
\addcontentsline{toc}{chapter}{INTISARI}
\paragraph*{}PT.Pundi Mas Berjaya adalah salah satu penyedia
solusi perangkat lunak di pasar global. PT. Pundi Mas
Berjaya memiliki pelanggan yang tersebar di beberapa negara,
dan pelanggan dari perusahaan yang bergerak di masing-masing
bidang yang berbeda. Dengan banyaknya pelanggan dengan asal
negara dan bidang yang bermacam-macam, tentunya
kebutuhan aplikasi perusahaan tersebut berbeda-beda pula.
Oleh karena itu PT.Pundi Mas Berjaya membutuhkan solusi untuk mengisolasi
aplikasi dan layanan yang dikembangkan dan di-\textit{host}, serta dapat di-\textit{deploy}
dengan cepat dan mudah.

\paragraph*{}Pada penelitian ini adalah membahas cara membangun
sebuah server LAMP dan email, dengan metode \textit{containerization} menggunakan Docker.
Sehingga memudahkan manajemen, dan \textit{deployment}
aplikasi.

\paragraph*{}Penelitian ini menghasilkan kesimpulan bahwa
dengan menggunakan Docker untuk manajemen dan
\textit{deployment} aplikasi menjadi lebih mudah dan tidak
memakan waktu yang lama dibandingkan dengan metode manual.\\

%\paragraph*{}LAMP \textit{stack} (Linux, Apache, MySQL, dan PHP)
%adalah platform pengembangan web open source populer yang
%banyak digunakan untuk membuat website dinamik dan aplikasi
%web. Sedangkan email server adalah sistem komputer yang
%digunakan untuk mengirim, menerima dan menyimpan pesan
%elektonik.
%
%\paragraph*{}LAMP \textit{stack} tersusun dari Linux sebagai
%OS, Apache sebagai web server, MySQL sebagai database, dan
%PHP sebagai \textit{server-side scripting}. Sedangkan untuk
%email server sendiri tersusun dari SMTP, POP, dan IMAP.
%Komponen-komponen perangkat lunak tersebut perlu disusun dan diatur
%sedemikian rupa sehingga dapat berinteraksi satu sama lain
%untuk membentuk sistem
%yang lengkap. Tahapan untuk pemasangan perangkat lunak
%tersebut cukup memakan waktu dan sulit untuk direplikasi
%jika sewaktu-waktu terdapat gangguan sistem atau pada saat
%melakukan pemasangan pada sistem yang lain. Selain itu tidak
%semua Sistem Operasi terdapat perangkat lunak yang
%diperlukan.
%
%\paragraph*{}Docker adalah salah satu platform untuk pengembangan,
%menyebarkan, dan menjalankan aplikasi yang dapat mengatasi
%permasalahan di atas. Dengan menggunakan docker semua
%library dan konfigurasi yang diperlukan dijadikan dalam satu
%\textit{container} yang dapat dijalankan di beragam Sistem
%Operasi.\\

\noindent
\textbf{Kata Kunci:} \textit{Container, Docker, LAMP, Email server}

\chapter*{ABSTRACT}\vspace*{2ex}
\addcontentsline{toc}{chapter}{ABSTRACT}

\paragraph*{} PT.Pundi Mas Berjaya is one of the leading software solution
providers in the global market. PT. Pundi Mas Berjaya has
customers spread across several countries, and customers
from companies engaged in each of the different fields. With
so many customers from various countries and fields, surely
they have applications requirements that varies too.
Therefore PT.Pundi Mas Berjaya needs a solution to isolate the application and their services, and can be deployed quickly
and easily.

\paragraph*{} In this research I tried to build a LAMP and
email server, with the containerization method using Docker.
So that makes applications management and deployment is
much more easier.

\paragraph*{} This study resulted in the conclusion that by
using Docker applications management and deployment become
easier and faster compared to manual method\\

%\paragraph*{} LAMP stack (Linux, Apache, MySQL, and PHP) is
%a popular platform that widely used for for dynamic website
%and web development. Email server is a system that used for
%sending, receiving, and stroing electronic mail.
%
%\paragraph*{}LAMP stack is composed from Linux as an OS,
%Apache for the web server, MySQL for the database, and PHP
%for server-side scripting. As for email server it composed
%from SMTP, POP, and IMAP. This components must be arranged
%and managed in such a way so that it can interact to each
%other to form a complete system. Process to setting-up these
%software is time consuming and difficult to replicate, if at
%any time there's a problem with the system, or when
%installing in other system. Furthermore not all Operating
%System have needed software.
%
%\paragraph*{}Docker is a platform for developing, shipping, and running
%applications. With using Docker all required library and
%configuration will be shipped in one container that can be
%run on many Operating System.\\

\noindent
\textbf{Keywords:} \textit{Container, Docker, LAMP, Email server}
\end{document}
