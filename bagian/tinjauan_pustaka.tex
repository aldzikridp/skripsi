\documentclass[../PROPOSAL_PRA_SKRIPSI_ALDZIKRI_DWIJAYANTO_PRATHAMA.tex]{subfiles}
\begin{document}
\bagian{Tinjauan Pustaka}
  \paragraph*{}Dalam mengimplementasikan MITM-Proxy untuk
  HTTPS \textit{filtering}, sebagai pedoman dan pembanding
  maka digunakan beberapa pustaka yang berkaitan dengan
  \textit{ad-blocking}. Pustaka yang digunakan sebagai
  rujukan antara lain:

  \paragraph*{} Dari penelitian yang dilakukan oleh
  \cite{uni2021}, yang berjudul \citetitle{uni2021}.
  Tujuan dari dilakukannya penelitian ini adalah untuk
  mengetahui rancangan sistem \textit{blocking} situs
  berbahaya dengan menggunakan \textit{pi-hole} berbasis
  docker dan \textit{openvpn}, dan menguji kemampuan
  \textit{blocking} situs dan performa dari \textit{pihole}
  dalam \textit{docker container}.

  \paragraph*{} \cite{yusoff2020}, Melakukan penelitian yang
  berjudul \textit{\citetitle{yusoff2020}}. Penelitian
  tersebut bertujuan untuk membangun server openvpn
  menggunakan Raspberry-pi dan menggunakan pi-hole sebagai
  sistem \textit{ad-block}. Selain itu penelitian ini juga
  memanfaatkan kemampuan pi-hole memahami \textit{regular
  expression} sebagai \textit{parental control} untuk
  mencegah diaksesnya situs-situs dewasa.

  \paragraph*{} Prosiding dengan judul
  \textit{\citetitle{sarath2020}}yang ditulis oleh
  \cite{sarath2020}. Tujuan dari penelitian tersebut adalah
  membuat sistem keamanan jaringan yang murah untuk bisnis
  kelas menengah ke bawah, dengan menggunakan Raspberry-pi
  sebagai perangkat kerasnya. Sedangkan di bagian perangkat
  lunak, digunakan Dnsmasq dan Squid proxy. Kedua perangkat
  lunak tersebut digunakan untuk memblokir domain dan url
  yang dianggap berbahaya.

  \paragraph*{} Prosiding oleh \cite{wahyudi2020}, dengan
  judul \citetitle{wahyudi2020}. Penelitian ini membahas
  bagaimana cara mengimplementasikan pi-hole di Ubuntu
  server sebagai DNS \textit{blocking} pada jaringan SMK TIK
  Darussalam Medan.

  \paragraph*{} Dari penelitian yang dilakukan oleh
  \cite{habibi2022} dengan judul \citetitle{habibi2022}.
  Penelitian ini membahas tentang pemasangan infrastruktur
  jaringan internet dengan menggunakan \textit{Point to
  Multipoint}, meminimalkan hilangnya paket data pada saat
  transmisi menggunakan RED dan PCQ dan menggunakan Sistem
  pi-hole di server untuk meminimalkan iklan yang muncul di
  protokol UDP.
\end{document}
